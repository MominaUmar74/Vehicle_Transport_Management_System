\documentclass[12pt,a4paper]{article}

\usepackage{graphicx}
\usepackage{geometry}
\usepackage{setspace}
\usepackage{array}

\geometry{
  left=30mm,
  right=30mm,
  top=25mm,
  bottom=25mm
}

\begin{document}

\begin{titlepage}
\centering

\vspace*{1cm}

{\Large\bfseries Vehicle Transport Management System (VTMS)\par}
\vspace{0.4cm}
{\large\bfseries Semester Project Proposal\par}

\vspace{0.8cm}

A project submitted in partial fulfilment of the requirements for the completion of\\
Software Engineering course

\vspace{1.5cm}

\includegraphics[width=5cm]{logo.png}

\vspace{1.5cm}

\begin{tabular}{c l @{\hspace{3cm}} l}
\textbf{} & \textbf{Name} & \textbf{Roll No} \\[0.3cm]
1 & Tahira Hanif & NUM-BSCS-2024-76 \\
2 & Momina Umer & NUM-BSCS-2024-36 \\
3 & Mubashir Aman & NUM-BSCS-2024-37 \\
\end{tabular}

\vfill

Department of Computer Science\\
Namal University Mianwali

\vspace{0.8cm}

28th December, 2025

\end{titlepage}
\setcounter{tocdepth}{3} % include subsections and subsubsections

\tableofcontents
\newpage

% ===============================
\section{Introduction}
Efficient vehicle transportation is very important in maintaining customer trust and en- suring a smooth and timely delivery of the vehicle. There are still many organizations who depend on traditional methods in transportation and shipment records and payments. The depend on this method often results in data loss, duplicate records, making trans- portation process late, billing errors, and also effects delivery tracking. To overcome this problem, this document provides the requirements for a Vehicle Transport Management System (VTMS). This system is designed to automate the vehicle purchasing, keeping the records of the vehicles and customers, track the shipment, and payment processing in order make an efficient process of transportation and improve the coordination among all the stockholders.


\subsection{Purpose}
\subsubsection{Overview of the Document}
This System Requirements Specification (SRS) document provides a detailed description of the VTMS functionalities, objectives, and requirements. The primary goal is to guide the development process and ensure alignment with the expectations of all stakeholders, including project sponsors, developers, and users.


\subsubsection{Purpose of the SRS}

The purpose of this document is to:

\begin{itemize}
    \item \textbf{Define Requirements:} Outline the functional and non-functional requirements for VTMS.
    \item \textbf{Guide Development:} Serve as a reference for the development team to ensure the successful implementation of features.
    \item \textbf{Align Stakeholders:} Clarify VTMS objectives for all stakeholders, ensuring mutual understanding and alignment.
\end{itemize}


\subsubsection{Intended Audience}

The SRS is designed to address the needs of diverse stakeholders involved in the development and deployment of VTMS.

\textbf{Stakeholders:}
\begin{itemize}
    \item \textbf{Transport Owners:} Ensure the system aligns with organizational objectives, such as operational efficiency, scalability, and customer satisfaction.
    \item \textbf{Project Managers:} Monitor development progress and ensure resource allocation aligns with project timelines and goals.
\end{itemize}

\textbf{Development Teams:}
\begin{itemize}
    \item \textbf{Software Engineers:} Use detailed functional and non-functional requirements as a roadmap for implementation.
    \item \textbf{System Architects:} Design scalable and secure infrastructure based on the specified requirements.
\end{itemize}

\textbf{End-Users:}
\begin{itemize}
    \item \textbf{Customers and Organizational Staff:} Ensure the system effectively supports vehicle booking, shipment tracking, billing, and management processes in a user-friendly and reliable manner.
\end{itemize}


\subsection{Scope}

\subsubsection{Product Identification}
Vehicle Transport Management System (VTMS) is a web-based computerized plat- form which is designed to manage and monitor vehicle transportation services efficiently. This system supports transport manager’s administrators and customers by automating the vehicle purchase, billing, record management and shipment tracking.


\subsubsection{Product Features}

\textbf{VTMS Will Provide:}
\begin{itemize}
    \item \textbf{User Services:} Enable users to register, browse available transport services, and place vehicle orders.
    \item \textbf{Shipment Tracking:} Provide shipment tracking, with location updates when shipments reach specific checkpoints.
    \item \textbf{Payment Management:} Allow users to make a 30\% advance payment and complete the remaining payment after delivery.
    \item \textbf{Transport Manager Support:} Assist transport managers in selecting vehicles, updating shipment status, and managing deliveries.
    \item \textbf{Administrative Controls:} Allow administrators to manage users, vehicles, schedules, orders, and system records.
    \item \textbf{Notifications and Records:} Generate invoices, send payment notifications to customers, and maintain transportation records.
\end{itemize}

\textbf{VTMS Will Not Include (Initial Release):}
\begin{itemize}
    \item Hardware-level GPS tracking devices.
    \item AI-driven advanced route optimization.
    \item A fully detailed user interface.
    \item A mobile application (initial release is web-based only).
\end{itemize}

\subsubsection{Goals and Benefits}
\begin{itemize}
    \item \textbf{Reliability:} Reduces errors and confusion caused by manual record-keeping and billing processes.
    \item \textbf{Customer Satisfaction:} Enhances customer trust through accurate tracking and timely delivery updates.
    \item \textbf{Scalability:} Lays the foundation for future expansion and feature enhancements.
\end{itemize}


\subsection{Overview}

\subsubsection{Document Structure}

This SRS document is organized as follows:
\begin{enumerate}
    \item \textbf{Introduction:} Describes the purpose, scope, and overview of the SRS.
    \item \textbf{Overall Description:} Details user personas, system environment, assumptions, and constraints.
    \item \textbf{Specific Requirements:} Provides in-depth functional and non-functional requirements.
    \item \textbf{System Features:} Enumerates core features like user profiles, skill listings, messaging, and feedback.
    \item \textbf{External Interface Requirements:} Explains interaction points such as APIs, user interface, and dependencies.
    \item \textbf{Other Non-Functional Requirements:} Covers performance, scalability, security, and reliability aspects.
    \item \textbf{Appendices and References:} Lists definitions, references, and additional resources.
\end{enumerate}

\subsection{Definitions, Acronyms, and Abbreviations}

\begin{table}[h]
\centering
\renewcommand{\arraystretch}{1.3}
\begin{tabular}{|p{4cm}|p{10cm}|}
\hline
\textbf{Term/Acronym} & \textbf{Definition} \\ \hline
VTMS & Vehicle Transport Management System \\ \hline
SRS & System Requirements Specification. \\ \hline
UI/UX & User Interface/User Experience, defining how users interact with the platform. \\ \hline
GitHub & A platform for version control and collaborative software development. \\ \hline
\end{tabular}
\end{table}


\subsection{References}






These references ensure the technical accuracy and alignment of VTMS development with industry standards and best practices.
\subsubsection{Standards}
\begin{enumerate}
    \item IEEE Std 830-1984: IEEE Recommended Practice for Software Requirements Specifications. IEEE, 1984.
    \item IEEE Std 1002-1987: IEEE Standard Taxonomy for Software Engineering Standards.
\end{enumerate}
\subsubsection{Technical Documentation}
\begin{enumerate}
    \item University Software Engineering Project Guidelines.
\end{enumerate}
\subsubsection{Development Tools}
\begin{enumerate}
    \item \LaTeX{} for documentation
    \item GitHub
    \item Figma
\end{enumerate}

% ===============================
\section{General Description}
This section provides a comprehensive overview of Vehicle Transport Management Sys- tem (VTMS), its context within related systems, major functionalities, user character- istics, and the constraints under which it operates. Rather than specifying detailed requirements, this section sets the stage for understanding those requirements by offering relevant background and descriptive elements.
 

\subsection{Product Perspective}
Vehicle Transport Management System is a standalone web application that operates independently to manage vehicle transportation operations. This system is not replace- ment of an existing software but it’s designed to automate the Transportation services like shipment record management, billing generation.


\subsubsection{Operational Constraints}
VTMS operates within defined environmental and development is subject to the following constraints.
1.	System Interfaces: The system uses standardized APIs for third-party services such as authentication, notifications (optional) and payment.
2.	User Interfaces: VTMS offers an intuitive web-based interface optimized for ac- cessibility.
3.	Hardware Requirements: VTMS can operate on any internet-enabled device with a modern web browser. No specialized hardware is necessary.

\subsection{Product Functions}

The Vehicle Transport Management System (VTMS) is designed to support vehicle exchange operations through a set of integrated functional components. These components work together to perform vehicle purchase, tracking, billing, and record management.

\begin{enumerate}
    \item \textbf{User Registration:} VTMS allows users to create accounts and log in using valid credentials. The system ensures that customers, transport managers, finance staff, and administrators have access only to their relevant features.
    
    \item \textbf{Vehicle Browsing and Purchase:} Customers can browse available transportation options and view details such as vehicle types, prices, and availability. They can select a specific vehicle and submit a purchase request.
    
    \item \textbf{Payment Processing:} The system supports advance and final payments. Customers pay 30\% of the bill as an advance during purchase and the remaining amount after delivery. Payment details are recorded, and notifications and receipts are sent to the customer.
    
    \item \textbf{Shipment Tracking:} Upon submission of a purchase request, VTMS generates a shipment record with a unique tracking ID. The system tracks the shipment from dispatch to final destination, ensuring smooth and accurate delivery.
    
    \item \textbf{Notifications and Communications:} Automated notifications are sent to administrators, customers, and other relevant stakeholders regarding payment confirmation, shipment dispatch, delivery, and order cancellations.
    
    \item \textbf{Reporting:} VTMS generates transport, shipment, and payment reports to help management and finance teams monitor operations and maintain accurate records.
\end{enumerate}

\subsection{User Characteristics}




\subsubsection{Educational Level and Background}
VTMS is designed to support users with different educational backgrounds and varying levels of technical understanding:

\begin{itemize}
    \item \textbf{Basic Users:} These include customers and staff with limited technical knowledge. The system provides simple and clear instructions to help them navigate the interface easily.
    
    \item \textbf{Intermediate Users:} Users with some technical background, such as transport staff, can access features like shipment tracking, status updates, and recording shipment details.
    
    \item \textbf{Advanced Users:} This group includes system administrators and technical staff with high technical proficiency. They manage user roles, identify and resolve system errors, and oversee all records related to vehicles and customers.
\end{itemize}
\subsubsection{Experience Levels}


The Vehicle Transport Management System (VTMS) is designed to support users with varying levels of experience in using computerized systems:

\begin{itemize}
    \item \textbf{Beginner Users:} Users with little or no prior experience with transport management software. VTMS provides an intuitive interface, guided workflows, and clear instructions to help them perform basic tasks such as vehicle browsing, booking, payments, and shipment tracking with minimal training.
    
    \item \textbf{Intermediate Users:} Users with moderate experience, such as transport staff or frequent customers, who can efficiently manage routine operations. They handle shipment updates, vehicle assignments, payment verification, and customer coordination using standard system features.
    
    \item \textbf{Expert Users:} Experienced users, including system administrators and senior transport managers, with extensive system knowledge. They utilize advanced features such as system configuration, user management, report generation, data monitoring, and overall system supervision.
\end{itemize}

\subsection{Assumptions and Dependencies}
The development, operation, and success of VTMS are influenced by various assumptions and dependencies. These elements play a crucial role in ensuring that the VTMS operates efficiently user needs. VTMS design and functionality rely on the following technical assumptions:
\subsubsection{Technical Assumptions}


The design and functionality of the Vehicle Transport Management System (VTMS) rely on the following technical assumptions:

\begin{itemize}
    \item \textbf{Stable Internet Connections:} VTMS assumes that users, including customers, transport managers, and administrators, have access to stable internet connections. While the system will be optimized for varying network conditions, real-time features such as shipment tracking, payment confirmation, and notifications may experience delays under poor connectivity.
    
    \item \textbf{Modern Web Browsers:} VTMS assumes that users will access the system through modern, supported web browsers such as Google Chrome, Mozilla Firefox, or Microsoft Edge. Using outdated or unsupported browsers may result in limited functionality, security vulnerabilities, or degraded user experience.
    
    \item \textbf{Secure Third-party Payment Gateway Integration:} VTMS assumes the availability and reliability of third-party payment gateways for processing advance and final payments. The accuracy, security, and availability of payment processing depend on these external services. Downtime or failure in the payment gateway may temporarily affect transaction processing.
    
    \item \textbf{Centralized Server and Database Availability:} The system assumes continuous availability of centralized servers and databases used to store user information, vehicle records, shipment data, and payment details. System performance and data accessibility depend on the stability and maintenance of the hosting infrastructure.
    
    \item \textbf{Real-Time Data Updates and Notifications:} VTMS assumes that real-time or near real-time data synchronization mechanisms will function reliably to update shipment status, delivery progress, and system notifications. Delays or failures in these mechanisms may impact timely information delivery to users.
\end{itemize}

\subsubsection{Operational Dependencies}


Several operational dependencies may impact the functionality and overall performance of the Vehicle Transport Management System (VTMS):

\begin{itemize}
    \item \textbf{Third-Party APIs:} VTMS relies on external services and APIs for essential operations, including payment processing, notification delivery (email or SMS), and potential map or location services for shipment tracking. The availability, reliability, and security of these services are critical for smooth system operation. Any changes, failures, or disruptions may affect transaction processing, communication with users, or shipment tracking features.
    
    \item \textbf{Compliance and Financial Services:} VTMS handles financial transactions, including advance and final payments, and relies on secure payment gateways and financial compliance mechanisms. The system must operate in accordance with applicable banking regulations and digital payment standards. Changes in payment service policies or regulatory requirements may require system updates to ensure continued compliance and uninterrupted service.
\end{itemize}

\subsubsection{User Behavior Assumptions}


The Vehicle Transport Management System (VTMS) operates under the following assumptions regarding user behavior:

\begin{itemize}
    \item \textbf{Accuracy and Integrity in Information Submission:} Users, including customers and organizational staff, are expected to provide accurate and truthful information during registration, vehicle booking, payment processing, and shipment updates. Correct entries are essential for reliable record keeping, shipment tracking, billing accuracy, and overall system effectiveness.
    
    \item \textbf{Responsible Use of Payment and Booking Features:} Users are assumed to responsibly use booking and payment functionalities, including making advance and final payments as defined and initiating cancellations only when necessary. Misuse of payment features or repeated incorrect transactions may affect system operations and financial reconciliation.
    
    \item \textbf{Engagement with System Notifications and Updates:} Users are expected to actively review system notifications related to payments, shipment status, and delivery confirmations. Timely attention to these updates is important for smooth transportation workflows and effective coordination between customers and transport management staff.
    
    \item \textbf{Compliance with Organizational Policies:} Users are assumed to adhere to organizational rules, refund policies, and VTMS usage guidelines. Misuse, policy violations, or unauthorized activities may require administrative intervention and could impact service continuity.
\end{itemize}


\subsubsection{Future Enhancements}

The Vehicle Transport Management System (VTMS) is designed with future expansion and continuous improvement in mind. The following provisions reflect potential enhancements planned for future versions of the system:

\begin{itemize}
    \item \textbf{Mobile Applications:} Future versions of VTMS may include mobile applications for Android and iOS platforms. Mobile access will allow customers and transport staff to manage bookings, receive real-time shipment updates, track deliveries, and receive notifications conveniently while on the move.
    
    \item \textbf{Advanced Tracking and Analytics Features:} VTMS may integrate advanced data analytics and intelligent tracking features to improve transportation efficiency. Potential enhancements include route optimization, delivery time predictions, and performance analysis to support better operational decision-making.
    
    \item \textbf{Increased User Load and System Scalability:} The system anticipates growth in users, vehicles, and shipment records over time. Future development may focus on improving scalability, server performance, load balancing, and database optimization to ensure continued reliability and availability.
\end{itemize}

\subsection{Design and Implementation Constraints}
\subsubsection{System Constraints}

The design and implementation of the Vehicle Transport Management System (VTMS) are influenced by several constraints:

\begin{itemize}
    \item \textbf{Initial Focus on Web-Only Implementation:} The first release of VTMS will be web-based only. Mobile applications and platform-specific clients are not included initially to ensure timely delivery and efficient use of available development resources. The web-based implementation provides sufficient functionality for managing vehicle bookings, payments, and shipment tracking.
    
    \item \textbf{Budget and Timeline Limitations:} VTMS development is constrained by a predefined budget and academic project timeline. Advanced features, such as AI-based route optimization, predictive analytics, and hardware-integrated tracking solutions, are deferred to future versions. The primary focus is on implementing core transportation management features reliably within the given timeframe.
    
    \item \textbf{Limited Team Size:} VTMS is being developed by a small project team, which may limit rapid feature expansion and large-scale scalability in the initial phase. The development effort will prioritize code quality, system reliability, and maintaina

\subsection{External Interface Requirements}

\subsubsection{User Interface Design}

The Vehicle Transport Management System (VTMS) shall provide a simple, intuitive, and user-friendly web-based interface to support all system users, including customers, transport managers, and administrators. The interface is designed to ensure ease of use, reduce user errors, and support efficient interaction with system functionalities.

\textbf{Design Visuals:}  
This includes the following artifacts:
\begin{itemize}
    \item Detailed wireframes and mock-ups for various pages of the website, including home pages, product detail pages, and checkout pages, showing layout, placement of elements, and visual design.
    \item Interactive prototypes or mock-ups demonstrating user interactions and transitions between pages.
\end{itemize}

\textbf{Interface Requirements:}
\begin{itemize}
    \item \textbf{Web-Based Interface:} VTMS shall be accessible through standard web browsers without requiring additional software installation. Supported browsers include Google Chrome, Mozilla Firefox, and Microsoft Edge.
    
    \item \textbf{Role-Based Interface Views:} Different layouts and functionalities shall be presented based on user roles. Customers, transport managers, and administrators will each have access to role-specific dashboards and controls relevant to their responsibilities.
    
    \item \textbf{Responsive Interface:} The interface shall be responsive, supporting multiple devices and screen sizes (desktop dashboards, mobile booking, and finance modules) for flexibility, scalability, and ease of maintenance.
    
    \item \textbf{Consistency and Navigation:} The interface shall maintain consistent layout, color schemes, and navigation controls across all pages. Clear menus, buttons, and labels shall help users easily locate system features.
    
    \item \textbf{Forms and Input Validation:} Structured input forms shall be provided for registration, booking, payments, and shipment updates. Inputs shall be validated, and clear error messages shall guide users in correcting invalid entries.
    
    \item \textbf{Filtering Options:} Clear filtering options shall be provided to help users quickly find their desired vehicles, enhancing the customer experience.
    
    \item \textbf{Shipment Tracking Display:} The interface shall include a dedicated view for shipment status and progress, providing real-time or near real-time updates in a clear format.
    
    \item \textbf{Notification and Feedback Messages:} Notifications related to booking confirmation, payment status, shipment updates, delivery confirmation, and refund processing shall be displayed clearly and concisely.
    
    \item \textbf{Accessibility and Usability:} The interface shall support users with minimal technical experience, using readable fonts, adequate contrast, and responsive layouts to improve accessibility and usability.
\end{itemize}


\subsubsection{Hardware and Computing Environment}

The Vehicle Transport Management System (VTMS) is designed to operate efficiently across commonly available hardware devices and computing environments.

\textbf{Mobile Devices (Future Support):}
\begin{enumerate}
    \item VTMS may support mobile devices such as Android and iOS smartphones in future versions, allowing users to access booking, payment, and shipment tracking features on the go.
    \item A responsive interface design will ensure compatibility with various screen sizes and resolutions.
\end{enumerate}

\textbf{Desktop and Laptop Systems:}
\begin{enumerate}
    \item The VTMS web application is optimized for desktop and laptop systems using modern web browsers such as Google Chrome, Mozilla Firefox, and Microsoft Edge.
    \item The system supports standard multitasking workflows, allowing users to access multiple system pages or browser tabs simultaneously.
\end{enumerate}

\textbf{Peripheral Devices:}
\begin{enumerate}
    \item VTMS supports standard input and output devices, including keyboards, mice, and display screens, required for system operation.
    \item Printer support is included for generating invoices, transport reports, and administrative records.
\end{enumerate}

\subsubsection{Software Interfaces}


The Vehicle Transport Management System (VTMS) interacts with multiple software components and external systems to ensure smooth and reliable operation.

\textbf{Authentication and Session Management:}
\begin{enumerate}
    \item VTMS provides secure user authentication using username/email and password-based login mechanisms.
    \item Session management ensures that authenticated users remain logged in during active sessions while preventing unauthorized access and session misuse.
\end{enumerate}

\textbf{Database Management Systems:}
\begin{enumerate}
    \item Structured system data, including user information, vehicle records, shipment details, and payment transactions, is stored in a centralized relational database.
    \item The database system ensures data consistency, integrity, and secure access for all system operations.
\end{enumerate}

\textbf{Payment Gateway Integration:}
\begin{enumerate}
    \item VTMS integrates with third-party payment gateway software to securely process advance and final payments.
    \item The system relies on these software interfaces for transaction authorization, confirmation, and payment status updates.
\end{enumerate}

\textbf{Notification Services:}
\begin{enumerate}
    \item VTMS interfaces with email and SMS services to deliver notifications related to booking confirmations, payment status, shipment updates, delivery confirmations, and refunds.
    \item These services ensure timely communication between the system and its users.
\end{enumerate}

\textbf{Mapping and Tracking Services (Future Support):}
\begin{enumerate}
    \item Future versions of VTMS may integrate with third-party mapping or tracking software to enhance shipment location visualization and route monitoring.
    \item These integrations will support improved tracking accuracy and delivery insights.
\end{enumerate}


\subsubsection{Communication Interfaces}

VTMS ensures smooth and reliable communication between control centers and user applications.

\textbf{Network Protocols:}
\begin{enumerate}
    \item Data is transmitted using HTTPS, ensuring secure communication between vehicles, servers, and client applications.
\end{enumerate}

\textbf{Error Recovery:}
\begin{enumerate}
    \item VTMS logs communication errors and performs automatic retries for failed data transmissions to reduce system downtime.
    \item A detailed error reporting system allows operators to report connectivity or synchronization issues directly to technical support.
\end{enumerate}

\subsection{Performance Requirements}


The Vehicle Transport Management System (VTMS) shall meet the following performance criteria:

\textbf{System Response Time:}
\begin{itemize}
    \item User actions (e.g., login, search vehicles, update shipment status) shall be completed within 2–3 seconds under normal load.
    \item Payment processing and confirmation shall complete within 5 seconds.
\end{itemize}

\textbf{Concurrent Users:}
\begin{itemize}
    \item VTMS shall support at least 500 concurrent users performing different actions (customers, administrators, finance staff, and transport staff).
    \item The system shall be scalable to handle increased user loads in the future without performance degradation.
\end{itemize}

\textbf{Database Performance:}
\begin{itemize}
    \item Database queries for vehicle browsing, shipment tracking, and payment history shall return results within 2 seconds for up to 10,000 records.
\end{itemize}

\textbf{Transaction Throughput:}
\begin{itemize}
    \item The system shall process at least 50 vehicle booking transactions per minute without errors.
    \item Payment transactions must maintain 100\% reliability with real-time confirmation.
\end{itemize}

\textbf{Availability and Uptime:}
\begin{itemize}
    \item VTMS shall maintain 99.5\% uptime, ensuring continuous availability for customers and staff.
    \item Maintenance operations should not affect live users or should be scheduled during low-traffic periods.
\end{itemize}

\textbf{Error Handling:}
\begin{itemize}
    \item In case of system failure (server crash, network issue), VTMS shall gracefully recover without data loss.
    \item Users shall receive informative error messages within 1–2 seconds.
\end{itemize}

\subsection{Software System Attributes}

\subsubsection{Security}
The Vehicle Transport Management System (VTMS) shall implement the following security measures:

\begin{itemize}
    \item \textbf{Password Protection:} All user passwords shall be encrypted to ensure secure storage and prevent unauthorized access.
    \item \textbf{Role-Based Access Control:} Access to system features and data shall be controlled based on user roles, ensuring that customers, transport staff, finance staff, and administrators only access relevant functionalities.
\end{itemize}
\subsubsection{Reliability}


The Vehicle Transport Management System (VTMS) shall ensure system reliability through the following measures:

\begin{itemize}
    \item \textbf{Data Protection:} The system shall prevent data loss under normal operation and in the event of system failures.
    \item \textbf{Regular Backups:} Regular backups of all critical data shall be maintained to enable recovery in case of unexpected failures.
\end{itemize}

\subsubsection{Maintainability}


The Vehicle Transport Management System (VTMS) shall be designed to support easy maintenance and updates:

\begin{itemize}
    \item \textbf{Modular Design:} The system shall be modular, allowing individual components to be updated or replaced without affecting other parts of the system.
    \item \textbf{Ease of Updates:} The system architecture shall facilitate straightforward implementation of new features, bug fixes, and improvements.
\end{itemize}

\subsection{Other Requirements}




\subsubsection{Database Requirements}

The Vehicle Transport Management System (VTMS) shall use a centralized database to ensure efficient and reliable data management:

\begin{itemize}
    \item \textbf{Centralized Database:} A single, centralized database shall store all system data, including user information, vehicle records, shipment details, and payment transactions.
    \item \textbf{Data Integrity:} The database shall enforce data integrity constraints to prevent corruption and ensure consistency across all records.
\end{itemize}
\subsubsection{Operational Requirements}
The Vehicle Transport Management System (VTMS) shall operate under the following conditions:

\begin{itemize}
    \item \textbf{Internet Connectivity:} Stable internet connectivity shall be required for all users to access system functionalities, including vehicle booking, payment processing, and shipment tracking.
\end{itemize}

\subsubsection{Site Adaptation Requirements}


The Vehicle Transport Management System (VTMS) shall support flexible deployment options:

\begin{itemize}
    \item \textbf{Deployment Options:} The system shall support both cloud-based deployment and local (on-premises) installation to accommodate different organizational needs.
\end{itemize}

% ===============================
\section{Appendices}

\subsection{Appendix A: Context Diagram}

\textbf{Description:}  
The context diagram illustrates the high-level interactions between the Vehicle Transport Management System (VTMS) and external entities, including customers, transport staff, administrators, and payment gateways (finance manager).

\begin{figure}[h!]
    \centering
    \includegraphics[width=1.0\textwidth]{untitled.png}
    \caption{High-level context diagram of VTMS}
    \label{fig:context_diagram}
\end{figure}


\subsection{Appendix B: Usecase Diagram}


\textbf{Description:}  
The use case diagram represents the functional interactions between system users and the Vehicle Transport Management System (VTMS). It highlights major system functionalities from the user perspective.

\begin{figure}[h!]
    \centering
    \includegraphics[width=0.9\textwidth]{usecase.png}
    \caption{Use Case Diagram of VTMS}
    \label{fig:usecase_diagram}
\end{figure}

\end{document}
